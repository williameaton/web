%%%%%%%%%%%%%%%%%%%%%%%%%%%%%%%%%%%%%%%%%
% Medium Length Professional CV - RESUME CLASS FILE
%
% This template has been downloaded from:
% http://www.LaTeXTemplates.com
%
% Original header:
% Copyright (C) 2010 by Trey Hunner
%
% Copying and distribution of this file, with or without modification,
% are permitted in any medium without royalty provided the copyright
% notice and this notice are preserved. This file is offered as-is,
% without any warranty.
%
% Created by Trey Hunner and modified by www.LaTeXTemplates.com
%
%%%%%%%%%%%%%%%%%%%%%%%%%%%%%%%%%%%%%%%%%
\documentclass{resume} 

\usepackage[left=0.40in,top=0.3in,right=0.75in,bottom=0.1in]{geometry} % Document margins
\usepackage{fontawesome}
\usepackage{times}
\usepackage{multirow}
\newcommand{\tab}[1]{\hspace{.2667\textwidth}\rlap{#1}}
\newcommand{\itab}[1]{\hspace{0em}\rlap{#1}}

% \begin{center}

% \end{center}
\name{Will Eaton} % Your name 

\address{\faGithub{   github.com/williameaton} \\   \faLaptop{   williameaton.github.io} \\  \faAt{   weaton@princeton.edu}}

\begin{document}

\begin{rSection}{Education}

{\bf Princeton University, USA} \hfill {\em (2021 - Present)} 
\\{ \textit {Graduate Student in Theoretical and Computational Seismology}} \\
Advisor: Professor Jeroen Tromp \\
Current GPA: 4.0  


{\bf University of Oxford, UK} \hfill {\em (2016 - 2021)} 
\\{ \textit {Integrated BA and MEarth Sci in Earth Sciences - First Class Honours }} \\
 Thesis: 'Seismic scattering on Mars, Earth, its moon and a supercomputer’\\
Advisor: Professor Tarje Nissen-Meyer

{\bf John Hampden Grammar School, High Wycombe, UK} \hfill {\em  (2009 - 2016)} 
\\ { \textit {A Levels (4 A*s), AS Level (1 A) and GCSE’s (10 A*s, 2 As) }} \hfill
%Minor in Linguistics \smallskip \\
%Member of Eta Kappa Nu \\
%Member of Upsilon Pi Epsilon \\


\end{rSection}

\begin{rSection}{Skills}

\begin{tabular}{ @{} >{\bfseries}l @{\hspace{6ex}} l }
Programming: \ & Fortran 90, Git, \LaTeX, MATLAB, Python, UNIX \\
Software \& Tools: & ArcGIS PRO, Adobe Illustrator, Paraview
\end{tabular}

\end{rSection}

% \begin{rSection}{Carrier Objective}
%  To work for an organization which provides me the opportunity to improve my skills and knowledge to grow along with the organization objective.
% \end{rSection}
%--------------------------------------------------------------------------------
%    Projects And Seminars
%-----------------------------------------------------------------------------------------------
\begin{rSection}{Research Experience}

{\bf Work Station x,Strange Place x} \hfill {\em June 20xx - April 20xx} 
\\{\textit{ Engineer}}
\\- xxx
\\- xxx



\end{rSection}

\begin{rSection}{Conference Proceedings}
\begin{tabular}{p{0.07\linewidth} | p{0.87\linewidth}}
\multirow{3}{*}{2022}  &  \textsc{\textbf{Eaton, W. P.}, Gharti, H. N., Tromp, J}., Seismic wave propagation in self-gravitating Earth models with 3D  heterogeneity. In \textit{AGU Fall Meeting 2022} (Chicago, IL, December 2022) \vspace{0.15cm} \\ &  \textsc{\textbf{Eaton, W. P.}, Haindl, C., Nissen-Meyer, T}., The transition from ballistic to diffuse wavefields on Earth, its Moon and Mars. In \textit{AGU Fall Meeting 2022} (Chicago, IL, December 2022) \vspace{0.15cm} \\ & \textsc{Gharti, H. N., \textbf{Eaton, W. P.}, Tromp, J}., Spectral-infinite-element Simulations of Seismic Wave Propagation in self-gravitating 3D Earth Models. In \textit{SSA Seismic Tomography: What comes next?} (Toronto, Canada, October 2022) \\
\end{tabular}

\end{rSection}



\begin{rSection}{Departmental Seminars}

\begin{tabular}{p{0.07\linewidth} | p{0.87\linewidth}}
\multirow{1}{*}{2022} & \textit{'Elasto-gravitational simulations on a realistic 3D Earth'}. UTIG Discussion Hour Seminar, University of Texas at Austin. Virtual, 28th November 2022. \\  
\end{tabular}



\begin{rSection}{Awards}

\begin{tabular}{p{0.07\linewidth} | p{0.87\linewidth}}
% 2021
\multirow{2}{*}{2021}& \textbf{Shell Prize} - \textit{Dept. of Earth Sciences, Oxford University} \\  & \hspace{0.7cm} Best overall performance in Earth Sciences Final Honours School.  \\
&  \textbf{Schlumberger Prize} - \textit{Dept. of Earth Sciences, Oxford University} \\ &\hspace{0.7cm} Best 4\textsuperscript{th} Year performance in Geophysics. \\ 
\\
% 2020
\multirow{3}{*}{2020}& \textbf{Gibbs Prize} - \textit{Dept. of Earth Sciences, University of Oxford} \\ & \hspace{0.7cm} Best undergraduate independent research (geological mapping) project. \\
& \textbf{Burdett-Coutts Prize} - \textit{Dept. of Earth Sciences, University of Oxford} \\ & \hspace{0.7cm} Best overall 3\textsuperscript{rd} Year performance in Earth Sciences Final Honours School. \\ 
& \textbf{University College Scholarship} - \textit{University College, University of Oxford} \\ & \hspace{0.7cm} Scholar status awarded in recognition of academic excellence. \\ 
\\
%2019 
\multirow{2}{*}{2019}& \textbf{Keith Cox Prize} - \textit{Dept. of Earth Sciences, University of Oxford} \\ & \hspace{0.7cm}Best 2\textsuperscript{nd} year fieldwork during Assynt fieldtrip, Scotland. \\
& \textbf{University College Scholarship} - \textit{University College, University of Oxford} \\ & \hspace{0.7cm} Scholar status awarded in recognition of academic excellence. \\ 
\\
% 2018 
2018& \textbf{International Seismological Centre Prize } - \textit{Dept. of Earth Sciences, University of Oxford} \\ & \hspace{0.7cm} Best 1\textsuperscript{st} Year student in Mathematics and Geophysics.

\\
% 2018 
2017 & \textbf{University College Exhibition} - \textit{University College, University of Oxford} \\ & \hspace{0.7cm} Exhibitioner status awarded in recognition of academic excellence. \\ 


\end{tabular}
\end{rSection}






\end{rSection}

\end{document}


