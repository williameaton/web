%%%%%%%%%%%%%%%%%%%%%%%%%%%%%%%%%%%%%%%%%
% Medium Length Professional CV - RESUME CLASS FILE
%
% This template has been downloaded from:
% http://www.LaTeXTemplates.com
%
% Original header:
% Copyright (C) 2010 by Trey Hunner
%
% Copying and distribution of this file, with or without modification,
% are permitted in any medium without royalty provided the copyright
% notice and this notice are preserved. This file is offered as-is,
% without any warranty.
%
% Created by Trey Hunner and modified by www.LaTeXTemplates.com
%
%%%%%%%%%%%%%%%%%%%%%%%%%%%%%%%%%%%%%%%%%
\documentclass{resume} 

\usepackage[left=0.40in,top=0.3in,right=0.75in,bottom=0.1in]{geometry} % Document margins
\usepackage{fontawesome}
\usepackage{times}
\usepackage{multirow}
\newcommand{\tab}[1]{\hspace{.2667\textwidth}\rlap{#1}}
\newcommand{\itab}[1]{\hspace{0em}\rlap{#1}}

% \begin{center}

% \end{center}
\name{Will Eaton} % Your name 
\address{\faGithub{   github.com/williameaton} \\   \faLaptop{   williameaton.github.io} \\  \faAt{   weaton@princeton.edu}}

\begin{document}
\centering 

\begin{rSection}{Education}

{\bf Princeton University, USA} \hfill {\em (2021 - Present)} 
\\{ \textit {Graduate Student in Theoretical and Computational Seismology}} \\
Advisor: Professor Jeroen Tromp \\
Current GPA: 4.0  


{\bf University of Oxford, UK} \hfill {\em (2016 - 2021)} 
\\{ \textit {Integrated BA and MEarth Sci in Earth Sciences - First Class Honours }} \\
Advisor: Professor Tarje Nissen-Meyer


{\bf John Hampden Grammar School, High Wycombe, UK} \hfill {\em  (2009 - 2016)} 
\\ { \textit {A Levels (4 A*s), AS Level (1 A) and GCSE’s (10 A*s, 2 As) }} \hfill
%Minor in Linguistics \smallskip \\
%Member of Eta Kappa Nu \\
%Member of Upsilon Pi Epsilon \\


\end{rSection}



\vspace{0.2cm}
\begin{rSection}{Research Experience and Projects}

{\bf Graduate studies in Theoretical and Computational Seismology} \hfill {\em (2020)} \\
{\textit{ \textbf{Elasto-gravitational numerical modelling on realistic, 3D Earth models}}} 

\begin{tabular}{p{0.5em}  p{0.925\linewidth}}
-&  Development of quasi-static, spectral-infinite-element modelling software for applications in glacio-isostatic adjustment and sea-level change.  \\
-&  Benchmarking of global-scale, elastic-wave-propagation simulations using normal-mode-summation codes.\\ 
-& Investigation and simulation of transient, seismically-induced gravity signals for earthquake early-warning systems and tsunami monitoring, and synthetic spectra of Earth's free oscillations for arbitrarily-complex, 3D Earth models.  \\ 
-& Supervised by Professor Jeroen Tromp (Princeton University) in collaboration with Professor Hom Nath Gharti (Queen's University)
\end{tabular}

\vspace{0.2cm}


{\bf Master's Thesis} \hfill {\em (2020 -  2021)} \\
{\textit{ \textbf{Seismic scattering on Mars, Earth, its moon and supercomputers}}} 

\begin{tabular}{p{0.5em}  p{0.925\linewidth}}
-& Investigating physical parameters facilitating a transition from ballistic to diffuse scattering behaviour of elastic
waves. \\
-& Numerical wave propagation through 3D heterogeneous media using AxiSEM3D.\\
-& Development and application of novel analytical techniques such as (moving-window) multi-scale entropy to
synthetic seismograms. \\
-& Analysis of Lunar Apollo and Martian InSight seismic data using these novel techniques to compare scattering
behaviour. \\
-& Supervised by Professor Tarje Nissen-Meyer. 
\end{tabular}


\vspace{0.2cm}
{\bf Batchelor's Extended Essay} \hfill {\em (2020)} \\
{\textit{ \textbf{Seismic heterogeneity and anisotropy in Earth’s inner core and the implications for inner core dynamics}}} 

\begin{tabular}{p{0.5em}  p{0.925\linewidth}}
-& Independent literature research project to produce 4000-word, review-paper-style extended essay.  \\
-& Skills gained in critical analysis of publications and synthesis/processing of publically-available data.
\end{tabular}

\vspace{0.2cm}

{\bf Undergraduate geological mapping project } \hfill {\em (2019 - 2020)} \\
{\textit{ \textbf{Geology and tectonic history of Saint-Chinian, Languedoc, France}}} 

\begin{tabular}{p{0.5em}  p{0.925\linewidth}}
-& Independent 6-week fieldwork project studying bedrock and collecting samples over 21 km\textsuperscript{2}, followed by
sample analysis culminating in 5000-word report. 
\end{tabular}

\end{rSection}

\newpage
\begin{rSection}{Skills}

\begin{tabular}{ @{} >{\bfseries}l @{\hspace{6ex}} l }
Programming: \ & Fortran, Git, \LaTeX, MATLAB, Python, UNIX \\
Software \& Tools: & ArcGIS PRO, Adobe Illustrator, Paraview, AxiSEM-3D, SPECFEM
\end{tabular}
\end{rSection}

\vspace{0.2cm}
\begin{rSection}{Conference Proceedings}
\begin{tabular}{p{0.07\linewidth} | p{0.87\linewidth}}
\multirow{3}{*}{2022}  &  \textsc{\textbf{Eaton, W. P.}, Gharti, H. N., Tromp, J}., Seismic wave propagation in self-gravitating Earth models with 3D  heterogeneity. In \textit{AGU Fall Meeting 2022} (Chicago, IL, December 2022) \vspace{0.15cm} \\ &  \textsc{\textbf{Eaton, W. P.}, Haindl, C., Nissen-Meyer, T}., The transition from ballistic to diffuse wavefields on Earth, its Moon and Mars. In \textit{AGU Fall Meeting 2022} (Chicago, IL, December 2022) \vspace{0.15cm} \\ & \textsc{Gharti, H. N., \textbf{Eaton, W. P.}, Tromp, J}., Spectral-infinite-element simulations of seismic wave propagation in self-gravitating, 3D Earth models. In \textit{SSA Seismic Tomography: What comes next?} (Toronto, Canada, October 2022) \\
\end{tabular}

\end{rSection}


\vspace{0.2cm}
\begin{rSection}{Departmental Seminars}

\begin{tabular}{p{0.07\linewidth} | p{0.87\linewidth}}
\multirow{1}{*}{2022} & \textit{'Elasto-gravitational simulations on a realistic 3D Earth'}. UTIG Discussion Hour Seminar, University of Texas at Austin. Virtual, 28th November 2022. \\  
\end{tabular}
\end{rSection}

\vspace{0.2cm}
\begin{rSection}{Awards}

\begin{tabular}{p{0.07\linewidth} | p{0.87\linewidth}}
% 2021
\multirow{4}{*}{2021}& \textbf{Shell Prize} - \textit{Dept. of Earth Sciences, Oxford University} \\  & \hspace{0.7cm} Best overall performance in Earth Sciences Final Honours School.  \\
&  \textbf{Schlumberger Prize} - \textit{Dept. of Earth Sciences, Oxford University} \\ &\hspace{0.7cm} Best 4\textsuperscript{th} Year performance in Geophysics. \\ 
\\
% 2020
\multirow{6}{*}{2020}& \textbf{Gibbs Prize} - \textit{Dept. of Earth Sciences, University of Oxford} \\ & \hspace{0.7cm} Best undergraduate independent research (geological mapping) project. \\
& \textbf{Burdett-Coutts Prize} - \textit{Dept. of Earth Sciences, University of Oxford} \\ & \hspace{0.7cm} Best overall 3\textsuperscript{rd} Year performance in Earth Sciences Final Honours School. \\ 
& \textbf{University College Scholarship} - \textit{University College, University of Oxford} \\ & \hspace{0.7cm} Scholar status awarded in recognition of academic excellence. \\ 
\\
%2019 
\multirow{4}{*}{2019}& \textbf{Keith Cox Prize} - \textit{Dept. of Earth Sciences, University of Oxford} \\ & \hspace{0.7cm}Best 2\textsuperscript{nd} year fieldwork during Assynt fieldtrip, Scotland. \\
& \textbf{University College Scholarship} - \textit{University College, University of Oxford} \\ & \hspace{0.7cm} Scholar status awarded in recognition of academic excellence. \\ 
\\
% 2018 
\multirow{2}{*}{2018} & \textbf{International Seismological Centre Prize } - \textit{Dept. of Earth Sciences, University of Oxford} \\ & \hspace{0.7cm} Best 1\textsuperscript{st} Year student in Mathematics and Geophysics.

\\
% 2017
\multirow{2}{*}{2017} & \textbf{University College Exhibition} - \textit{University College, University of Oxford} \\ & \hspace{0.7cm} Exhibitioner status awarded in recognition of academic excellence. \\ 


\end{tabular}
\end{rSection}

\vspace{0.2cm}
\begin{rSection}{Professional Associations and Memberships}

{American Geophysical Union} \hfill {\em January 2021 - Present} \\
{Seismological Society of America} \hfill {\em February 2021 - Present} 
\end{rSection}
\end{document}